\section{Dialogue with the robot}

\subsection{Client-Server}

The robot uses a TCP protocole for communication, this means that the robot communicate with potential client or sever (depending on how you set up the robot) in a connected way. The main advantage of this type of communication is that we are sure that the message send to or by the robot will arrive, but the main disavantadge is that the functions to read data are blocking. It means that if there is any problem on the network or if one of the two part are lagging or fail to communicate the whole program could freeze.

In our case the robot is set as a server and the program as a client. At first the program was digned to work the other way around, the program as a server and the robot as a client. This way would have been the simplest one because it would have used the blocking call of the TCP communication to handle the delays for the robot to move.
The workflow would have been the following :
\begin{itemize}
    \item The robot connect to the program;
    \item The program send command to the robot and wait for it to receive that command;
    \item The robot receive the command and carry out the movement;
    \item While the robot is executing the command the program can send a new one;
    \item The new command is here waiting for the robot to receive it (the program will wait due to the blocking call of the TCP protocol);
    \item Once the robot is done, it can read the new data or wait for some new one.
    \item We repeat this until the end.
\end{itemize}

This way the delay for the robot to execute is handle through the TCP protocol. But unfortunately their is no way to update a program quickly into the robot and since the letter library is quite big it was complicated to do it that way, thus we reverse it to use the robot as a server and the program as a client.

The problem now is to handle the delay ecause when the robot is set as a server, each new command will replace the previous one thus in order to draw complete letter there is a need to set a delay. The delay was set within the Communicator class and made to be more than enough to give the robot time to draw each part of the letter. This way is not prefect because if there is any delay within the robot drawing, the new command could be sent too earl and then this would mess up the drawing of the letter.

\subsection{Commands}

In order to communicate with the robot the program needs to send command following a precise syntaxe.
\begin{itemize}
  \item The data must be surrounded by bracket pair
  \item Every data unit must be split by comma
  \item Must add '\n' as the terminator
\end{itemize}

But there was trouble with the communication thus it was added a '\r' to the end of the signal. But the main problem was with the size of the buffer that was sent. The buffer was a hundred char long and was sent completely each time. But it seems that the robot needs to receive the exact command, i.e nothing more at the end, no space no nothing, or there is a bug and the robot doesn't recognise the following commands. 
