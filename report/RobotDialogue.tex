\section{Dialogue with the robot}

\subsection{Client-Server}

The robot uses a TCP protocole for communication, meaning that the robot communicate with potential client or sever (depending on how the robot is set up) in a connected way. The main advantage of this type of communication it is sure that the message sent to or by the robot will arrive, but the main disavantage is that the functions to read data are blocking (non-asynchronous). It means that if there is any problem on the network or if one of the two part is delayed or fail to communicate the whole program could freeze.

In our case the robot is set as a server and the program as a client. At first the program was designed to work the other way around, the program as a server and the robot as a client. This way would have been the simplest one because it would have used the blocking call of the TCP communication to handle the delays for the robot to move.
The workflow would have been the following :
\begin{itemize}
    \item The robot connects to the program;
    \item The program sends command to the robot and waits for it to receive that command;
    \item The robot receives the command and carries out the movement;
    \item While the robot is executing the command the program can send a new one;
    \item The new command is here waiting for the robot to receive it (the program will wait due to the blocking call of the TCP protocol);
    \item Once the robot is done executing the command, it can read the new data or wait for some new one.
    \item This is being repeated until the end.
\end{itemize}

This way the delay for the robot to execute is handled through the TCP protocol. But unfortunately there is no way to update a program quickly into the robot and since the letter library is quite big, it was complicated to do it that way. Thus the robot was used as a server and the program as a client.

The problem now is to handle the delay because when the robot is set as a server, each new command will replace the previous one, thus in order to draw complete letters there is a need to set a delay. The delay was set within the Communicator class and made to be more than enough time to let the robot draw each part of the letter. This way is not prefect because if there is any delay within the robot drawing, the new command could be sent too early and this would disturb the drawing of the letter.

\subsection{Commands}

In order to communicate with the robot the program needs to send command following a precise syntax.
\begin{itemize}
  \item The data must be surrounded by bracket pair;
  \item Every data unit must be seperated by a comma;
  \item There must be '\textbackslash n' as the terminator.
\end{itemize}

However there were troubles with the communication thus a '\textbackslash r' was added at the end of the signal (before the \textbackslash n). But the main problem was with the size of the buffer that was sent. The buffer was a hundred char long and was entirely sent each time. It also seemed that the robot needed to receive the exact command, i.e nothing more at the end, or there was a bug and the robot did not recognise the following commands.
