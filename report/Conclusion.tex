\section{Conclusion}
This application goal was to implement writing and drawing with a robot arm, controlled by user gestures on a Leap Motion. To be able to control the writing, a set of gestures has been implemented. In order to have an application as much user-friendly as possible, those gestures used are static and very simple.\\
As of now, the application implements both the writing and the drawing functions successfully.\\

However, some improvements are possible on this application. For example, in our alphabet, we have gestures that do not correspond to any letter. Those gestures could be used to switch between upper-case and lower-case letters.\\
Since the application uses a library containing all movements for writing the letters, it could be possible to develop other libraries that would allow the user to use non-Latin character (such as Chinese characters, the Korean or the Arabic alphabet, etc..).\\

As the application handles dynamic movements from the user with the drawing function, one possible improvement could be implementing the English Sign Language. The user would then be able to write without having to learn another sign alphabet.\\
Moreover, in the drawing function, implementing gestures allowing the user to raise and lower the robot arm (to stop or start drawing), would greatly improve the user-friendliness of the application.\\

Finally, as described earlier, there is some delay applied when communicating with the robot. That delay could be optimized in order to attain a faster writing of the robot and avoid frustration from the user.\\

To conclude, the development of this application was focused on its use. The final result is functional and user-friendly. However, some improvements could be made in order to make the application more diverse and attractive.
